\documentclass[a4paper]{article}

%% Language and font encodings
\usepackage[english]{babel}
\usepackage[utf8x]{inputenc}
\usepackage[T1]{fontenc}

%% Sets page size and margins
\usepackage[a4paper,top=3cm,bottom=2cm,left=3cm,right=3cm,marginparwidth=1.75cm]{geometry}

%% Useful packages
\usepackage{amsmath}
\usepackage{amsfonts}
\usepackage{graphicx}
\usepackage[colorinlistoftodos]{todonotes}
\usepackage[colorlinks=true, allcolors=blue]{hyperref}

\title{Judson's Abstract Algebra: Chapter 9}
\date{}

\begin{document}
\maketitle

\section*{1}

Prove that $\mathbb{Z} \cong n \mathbb{Z}$ for $n \neq 0$.

\vspace{\baselineskip}

Let $n \in \mathbb{Z}, n \neq 0$. Consider the function $f : \mathbb{Z} \rightarrow n \mathbb{Z}$

$$f(x) = nx.$$

Let $x,y \in \mathbb{Z}$. Note that 

$$f(x+y) = n(x+y) = nx + ny = f(x) + f(y).$$

Assume $f(x) = f(y)$. Then

$$f(x) = f(y)$$
$$nx = ny$$
$$x = y$$

so $f$ is injective. Note that $f$ is surjective by definition. Since $f$ is a bijection and preserves the group operation $f$ is an isomorphism.


\section*{2} 

Prove that $\mathbb{C}^*$ is isomorphic to the subgroup of $GL_2(\mathbb{R}$ consisting of matrices of form

$$
  \begin{pmatrix}
    a & b \\
    -b & a
  \end{pmatrix}.$$
  
We will call the subgroup described above $M$. Consider a function $f : \mathbb{C}^* \rightarrow M$ defined by 

$$f(a+bi) = \begin{pmatrix}
    a & b \\
    -b & a
  \end{pmatrix}.$$
  
Let $a,b,c,d \in \mathbb{R}$ where $ab \neq 0$ and $cd \neq 0$. First note that the function preserves the group operation since

\begin{align*}
f((a+bi)(c+di)) &= f(ac-db + (db + da)i) \\
&= \begin{pmatrix}
    ac-db & cb + da \\
    cb + da & ac - db
  \end{pmatrix} \\
&= = \begin{pmatrix}
    a & b \\
    -b & a
  \end{pmatrix}
  = \begin{pmatrix}
    c & d \\
    -d & c
  \end{pmatrix} \\
&= f(a+bi)f(c+di)
\end{align*}

Note that $f$ is surjective because $a,b$ are arbitrary real numbers. Notice that $f$ is injective because of element-wise equality of matrices.


\end{document}