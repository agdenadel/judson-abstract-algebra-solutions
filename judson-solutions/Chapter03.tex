\documentclass[a4paper]{article}

%% Language and font encodings
\usepackage[english]{babel}
\usepackage[utf8x]{inputenc}
\usepackage[T1]{fontenc}

%% Sets page size and margins
\usepackage[a4paper,top=3cm,bottom=2cm,left=3cm,right=3cm,marginparwidth=1.75cm]{geometry}

%% Useful packages
\usepackage{amsmath}
\usepackage{amsfonts}
\usepackage{graphicx}
\usepackage[colorinlistoftodos]{todonotes}
\usepackage[colorlinks=true, allcolors=blue]{hyperref}

\title{Judson's Abstract Algebra: Chapter 3}
\date{}

\begin{document}
\maketitle

\section{}

Suppose $3x \equiv 2 \mod 7$. Then $x \in \{z: z = 3 + 7n, n \in \mathbb{Z}\}$.

\vspace{\baselineskip}

Suppose $5x + 1 \equiv 13 \mod 23$. Then $x \in \{z: z = 7 + 23n, n \in \mathbb{Z}\}$.

\vspace{\baselineskip}

Suppose $5x + 1 \equiv 13 \mod 26$. Then $x \in \{z: z = 18 + 26n, n \in \mathbb{Z}\}$.

\vspace{\baselineskip}

Suppose $9x \equiv 3 \mod 5$. Then $x \in \{z: z = 2 + 5n, n \in \mathbb{Z}\}$.

\vspace{\baselineskip}

Suppose $5x \equiv 1 \mod 6$. Then $x \in \{z: z = 5 + 6n, n \in \mathbb{Z}\}$.

\vspace{\baselineskip}

Suppose $3x \equiv 1 \mod 6$. There are no solutions.

\section{}

\section{}

\section{}

\section{}

\section{}

\section{}

\section{}

\section{}

\section{}

Prove that the set of matrices of the form
$$
  \begin{pmatrix}
    1 & x & y \\
    0 & 1 & z \\
    0 & 0 & 1
  \end{pmatrix}
$$

is a group under matrix multiplication. This group, known as the \textit{Heisenburg group}, is important in quantum physics. Matrix multiplication in the Heisenberg group is defined by

$$
  \begin{pmatrix}
    1 & x & y \\
    0 & 1 & z \\
    0 & 0 & 1
  \end{pmatrix}
  \begin{pmatrix}
    1 & x' & y' \\
    0 & 1 & z' \\
    0 & 0 & 1
  \end{pmatrix}
=
  \begin{pmatrix}
    1 & x + x' & y + y'+ xz' \\
    0 & 1 & z + z' \\
    0 & 0 & 1
  \end{pmatrix}
$$

The proof that this set of matrices forms a group is straightforward. Associativity follows from the associativity of matrix multiplication. The identity element is the usual identity matrix. Each matrix in the set is upper triangular and has an inverse (if this is not obvious, then the matrices can be easily put in reduced row echelon form).



\end{document}